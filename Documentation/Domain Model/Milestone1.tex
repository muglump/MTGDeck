\documentclass{article}
\usepackage{fancyhdr}
\usepackage[margin = 0.5in]{geometry}
\usepackage{setspace}
\usepackage{hyperref}
\usepackage{glossaries}
\usepackage{graphicx}
\pagenumbering{arabic}
\onehalfspacing
\pagestyle{fancy}
\fancyhead[L]{MTG Deck Builder}
\fancyhead[C]{Domain Track Milestone 1}
\fancyhead[R]{4-18-2013}

\begin{document}
	\begin{titlepage}
		\vspace*{\fill}
		\hrulefill
		\begin{center}
			\section*{Domain Model Milestone 1}
			\subsection*{CSSE 376}
			\subsection*{Thomas Morris, David Savrda, Alexander White}
			4-18-2013
		\end{center}
		\hrulefill
		\vspace*{\fill}
	\end{titlepage}
	\newpage
	\section*{What does quality mean in terms of mobile development?}
		In terms of mobile development quality is a measure of the functional and aesthetic utility to the consumer.
	\section*{What are the important factors in quality assurance for mobile development?}
		For mobile development there are several key factors that determine if an application is considered of good quality.
		\begin{itemize}
			\item Stable and reliable - Mobile applications are generally more difficult to test, as there are frequently more variables then in traditional software development. To ensure that an application is both stable and reliable in additional to traditional testing several other conditions. Conditions such as: no wifi, no cellular data, air-plane mode, device versions, OS versions, screen sizes, etc. Additionally mobile applications must be especially cautious of over-using system resources.
			\item Platform Consistency - For mobile development it is especially important to follow the guidelines of the platform you are developing for. This is important because the platform has a large impact on the UI and the user experience as a whole. When using a certain mobile device the user expects a standard device layout, and navigation controls and venturing outside of that can cause user discomfort.
			\item Loads fast - Form mobile applications fast load times are especially important, as user want rapid results. This means that an application should fully load within at most under 5 seconds or most users will switch to something else.
			\item No Blocking UI - In mobile development UI is undoubtedly the driving force, and what the user is expecting. In more traditional applications a user may wait for several seconds if a program hangs, however in mobile applications any hangs are usually regard a program freezes and results in the program being terminated by the user.
		\end{itemize}
	\section*{What processes are currently used in mobile development to assure quality?}
		From visiting several companies offering mobile application testing to developers there are several key steps that seem universal to mobile application quality assurance.
		\begin{itemize}
			\item Functional Testing - Assuring that the application actually does what it was built to do is one of the most key parts of quality assurance for mobile applications. This step generally includes not only ensure the application functions as required but also that it does not consume an inordinate amount of system resources doing it.
			\item Configuration testing - This is an extremely important step in assuring that your mobile application is of good quality. By testing you application in the various operating systems, devices, and other conditions that it will be available on, you ensure that the application will be equally usable for all users.
			\item Usability Testing - Another extremely important step in assuring that you mobile application meets the quality that you desire of it. By including potential user level individuals in your testing process you can identify areas that may be confusing to your potential user base so that they may be corrected. Additionally users may find cases that you never would have tested. 
		\end{itemize}
	\section*{What processes do you suggest?}
		\begin{itemize}
			\item \textbf{Functional testing} - This is required for the same reason as it is in standalone software development. However, even more so, functional testing and measuring resources used by the system is necessary because of the hardware limitations present on mobile devices.
			\item \textbf{Configuration testing} - This is required too, even when developing for iOS system. Both iOS and Android development specifications say that you must test your on a real device before being released. Apple suggests that you run configuration tests for, at a minimum "all devices you have available. Ideally, test the app on all devices and iOS versions you intend to support." Similarly, while Android doesn't have nearly as stringent requirements to develop and release an application as the iOS store does, they still suggest testing it on at least one device. Similarly to iOS, the developer should test every version of Android that they intend to support.
			\item \textbf{Usability Testing}- Usability testing is also extremely in regards to mobile application development. However there is no real distinction between developing for a mobile environment verses standard software in regards to Usability Testing 
		\end{itemize}
	\section*{Are there any regulations/standards?}
		There are many regulations and standards governing mobile development, most which vary depending on the particular mobile environment being developed for. Some examples of standards that all iOS applications include: Returning to the applications previous UI state at launch time, applications must be tuned for performance(not draining battery or impact resources), and including some specific resources unnecessary to the running of the application (such as the icon that represents the app on the main screen).
	\section*{What metrics are used for mobile development?}
		\begin{itemize}
			\item Behavioral analytics - Page tagging solutions with companies that measure mobile websites and apps.
			\item Attitudinal analytics - Companies survey people while they are operating the mobile devices for information about mobile experience.
			\item Average number of keystrokes, and clicks - Should be more applicable in mobile
			\item Search function usage - Used constantly in mobile transaction
			\item Drive tests - Extensive measurement of the quality of service in North America
			\item Monitoring Network Signaling - Count the amount of mobile subscribers and determine market share in 86 U.S. markets
		\end{itemize}
	\section*{Metrics suggested to be used}
		\begin{itemize}
			\item Wi-Fi access - Should be consistently high and increasing due to the amount of people with mobile devices
			\item Report page - The number of reports provided from users in a given month
		\end{itemize}
	\section*{Success and failures in Mobile Development}
		\subsection*{Successes}
			\begin{itemize}
				\item Mobile Device solves problems - The user gains benefits from using this mobile device
				\item Focus on one thing and do it well - When one thing is focused on, the project can have its full effect and effort focus on accomplishing the one main goal
				\item Mobile apps - Mobile applications that can be downloaded and engages the user
			\end{itemize}
		\subsection*{Failures}
			\begin{itemize}
				\item Apps without flexibility - App doesn't have product change, typo, and content addition or removal
				\item No innovation - A new mobile device has minimal to no new additional features that set them apart from competition
				\item Too much creativity - The mobile device is difficult for the user to understand and/or software fails to perform due to too many features
			\end{itemize}
	\section*{Common problems with Quality Assurance in Mobile Development}
		\begin{itemize}
			\item Usability - Implied restrictions on interfaces for user interaction in comparison to a normal desktop
			\item Device heterogeneity - Lack of accepted application level models and maintaining multiple device dependent versions is labor intensive
			\item Network limitation - Resource based efficiency and Time based efficiency
		\end{itemize}
	\section*{Sources}
		\begin{itemize}
			\item \url{http://www.google.com/url?sa=t&rct=j&q=&esrc=s&source=web&cd=3&ved=0CEkQFjAC&url=http%3A%2F%2Fwww.researchgate.net%2Fpublication%2F2886929_Quality_Attributes_in_Mobile_Web_Application_Development%2Ffile%2F32bfe50f19680bd614.pdf&ei=U_2CUZ7eBfDyyAHmv4DQCg&usg=AFQjCNEccYgcpdvJVFOBPKXZ5hnjxMCNYA&sig2=mkjTkm20L_AeqQFjN9qRgg}
			\item \url{http://www.computerworld.com/s/article/9235570/Scot_Finnie_5_tips_for_developing_successful_mobile_apps}
			\item \url{http://www.allvoices.com/contributed-news/14127039-reasons-for-failure-of-apps-in-the-mobile-market}
			\item \url{http://www.tripwiremagazine.com/2013/04/app-development-tutorials.html}
			\item \url{http://www.on3solutions.com/blog/4-essential-tips-and-the-secret-to-success-in-mobile-software-development/}
			\item \url{http://newstex.com/2013/04/15/android-and-ios-user-data-key-to-mobile-development-success/}
			\item \url{http://www.nielsen.com/us/en/nielsen-solutions/nielsen-measurement/nielsen-mobile-measurement.html}
			\item \url{http://online-behavior.com/analytics/mobile-shopping}
			\item \url{http://mobilegovwiki.howto.gov/Mobile+Analytics+Usage+%26+Other+Metrics#Mobile%20Performance%20Metrics}
			\item \url{https://developer.apple.com/library/ios/#documentation/iPhone/Conceptual/iPhoneOSProgrammingGuide/Indroduction/Indroduction.html}
			\item \url{http://developer.android.com/guide/components/index.html}
			\item \url{http://dotnet.dzone.com/articles/what-makes-app-good-app-10}
			\item \url{https://www.google.com/url?sa=t&rct=j&q=&esrc=s&source=web&cd=5&cad=rja&ved=0CGoQFjAE&url=http%3A%2F%2Fwww.researchgate.net%2Fpublication%2F2886929_Quality_Attributes_in_Mobile_Web_Application_Development%2Ffile%2F32bfe50f19680bd614.pdf&ei=vf5vUcjILuPX2QWmtYDYBw&usg=AFQjCNEccYgcpdvJVFOBPKXZ5hnjxMCNYA&sig2=rYDoZVHQX_IyrCbW7-N8aw&bvm=bv.45368065,d.b2I}
			\item  \url{http://meetup.baltimoremobile.org/events/79898512/?eventId=79898512&action=detail}
			\item \url{http://multichoiceapps.com/quality-assurance.html}
			\item \url{http://www.indianic.com/quality-assurance-process-mobile.html}
		\end{itemize}
		
\end{document}